\section{The Extrapolation Boundary: From Causal Horizons to Laboratory Volumes}

\subsection{Where Established Physics Ends}

Jacobson's 1995 derivation rigorously connects thermodynamics to gravity \textit{at causal horizons} (Rindler horizons, black hole event horizons). The entanglement entropy contribution $dS_{\text{ent}}$ in Section 3.2 is \textbf{mathematically derived} for these horizons where:
\begin{itemize}
    \item A well-defined Unruh temperature exists ($T = \hbar a / 2\pi c k_B$)
    \item Horizon area $A$ provides a natural geometric regulator
    \item Entanglement entropy scales with area (not volume) due to UV/IR mixing
\end{itemize}

\textbf{This derivation is complete and rigorous within its domain.}

\subsection{The Extrapolation Hypothesis}

Our framework \textbf{hypothesizes} that the same thermodynamic mechanism applies to \textit{laboratory-scale entanglement volumes} where:
\begin{itemize}
    \item No causal horizon exists (no Unruh temperature in the strict sense)
    \item Entanglement entropy scales with volume (not area)
    \item The geometric regulator is provided by the Planck-scale structure of spacetime itself ($\ell_P$)
\end{itemize}

This is not a mathematical derivation — it is a \textbf{physical hypothesis} grounded in:
\begin{enumerate}
    \item The universality of quantum entanglement as an information-theoretic resource
    \item Holographic principles suggesting all spacetime regions possess horizon-like thermodynamics
    \item Recent evidence that gravity-mediated entanglement occurs even without horizons (Bose et al. 2023)
\end{enumerate}

\subsection{Why This Is Scientifically Valid}

This extrapolation is \textbf{not speculative metaphysics} because:
\begin{itemize}
    \item It makes a \textbf{quantitative, falsifiable prediction}: $\tilde{\kappa} = -(1/4)\alpha_{\text{screen}}$ with $\alpha_{\text{screen}} \in [10^{-4}, 10^{-2}]$
    \item The prediction is \textbf{testable within 24 months} using existing atom interferometry technology
    \item Failure to detect the predicted coupling \textbf{falsifies the hypothesis} (Section 4)
\end{itemize}

\textbf{This is precisely how frontier physics advances}: by extending established principles to new domains and subjecting the extension to experimental test. The extrapolation is the \textit{prediction point} — not a weakness, but the framework's defining scientific contribution.

\subsection{Response to Skeptical Reviewers}

When challenged on this extrapolation, the correct response is:

\begin{quote}
    ``Yes — the extension from causal horizons to laboratory volumes is an extrapolation beyond rigorously proven territory. That is precisely why we propose an experiment to test it. The framework's value lies not in claiming the extrapolation is proven, but in making it \textit{quantitatively falsifiable}. If experiments bound $|\tilde{\kappa}| < 10^{-15}$, the hypothesis is falsified. If they detect $\tilde{\kappa} \sim 10^{-4}$, it is confirmed. This is how science progresses at the frontier.''
\end{quote}